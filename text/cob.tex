\chapter{Cache-oblivious B-trees}

\section{Cache-aware B-trees}
% TODO: usual => good? best?
In the external memory model, the usual way of storing sorted dictionaries
is using a B-tree. A B-tree is a generalization of balanced binary search
trees, in which nodes keep up to $b$ key-value pairs in sorted order.
An inner node of a binary search tree with key $X$ contains two pointers
pointing to children with keys $< X$ and $\geq X$. The B-tree generalizes
this by having an inner node with $K_1,\ldots K_k$ keeping $k+1$ pointers
to children with keys $< K_1$, $K_1\cdots K_2-1$, \dots, $> K_k$.

The process of finding a key-value pair in a B-tree starts at the root node
and walks down the tree using a generalization of binary search. Insertions
similarly first find the right place to put the new key-value pair, and
then walk back up the tree. If an overfull node (with more than $b$ keys)
is encountered, it is split to two nodes of $b/2$ keys and a pointer to
the new node is inserted into the parent. Deletions employ a similar procedure,
merging underfull nodes (with less than $b/2$ keys) with their siblings.

Thus, updates to the B-tree keep the number of keys within all nodes between
$b/2$ and $b$, so the depth of the tree is kept between $\log_b N$ and
$\log_{b/2} N$. The \textsc{Insert}, \textsc{Delete} and
\textsc{Find} operations therefore run in $\Theta(\log_b N)$ time.
The \textsc{FindNext} and \textsc{FindPrevious} also run in $\Theta(\log_b N)$ time,
but if they are invoked after a previous \textsc{Find}, we can keep a pointer
to the leaf containing the key and accordingly adjust it to find the next
or previous key, which runs in $\O(1)$ amortized time.
Thus, B-trees also allow scanning a contigous range of keys of size $n$
in $\Theta(\log_b N + n)$ time.

If we choose the parameter $b$ to be $\Theta(B)$, we obtain a bound of
$\Theta(1+\log_B N)$ block transfers for all operations, which can
be easily shown to be optimal in the comparison model.
A contigous range of $n$ keys can be
read using $\Theta(\log_B N+n/B)$ block transfers.
Thus, the B-tree has optimal performance in the comparison model
if the block size is known.

The \textit{cache-oblivious B-tree} introduced by \cite{demaine00}
is a data structure which gives similar bounds in a cache-oblivious
setting.

TODO: high-level popis
TODO: indirection and merge/splits

\section{The Van Emde Boas layout}
One of the building blocks of the cache-oblivious B-tree is the Van Emde Boas
layout (named after the Van Emde Boas priority queue, which uses similar ideas).
The Van Emde Boas layout is a way of mapping the nodes of a full binary
tree of height $h$ to indices $0\ldots 2^h-2$. Other layouts of full binary
trees include the BFS or DFS order.

The advantage of storing a full binary tree in the Van Emde Boas layout
is that is lets us read the sequence of keys from the root to any leaf
using $\Theta(1+\log_B N)$ block transfers, which matches the \textsc{Find}
cost of B-trees without the need to know $B$ beforehand.
In contrast, the same operation would cost $\Theta(1+\log N-\log B)$ block
transfers in the DFS or BFS order.

The Van Emde Boas layout is defined recursively. To find the Van Emde Boas layout
of a full binary tree of height $h$, we split the tree to bottom subtrees
of height $\lhfloor h-1 \rhfloor$ and one top subtree of height $h - \lhfloor
h-1 \rhfloor$.
The subtrees are recursively aligned in the Van Emde Boas layout and then laid
out - first the top tree, then the bottom trees in BFS order. The Van Emde Boas
layout of a one-node tree is trivial.

\begin{figure}
\centering
\tikzset{
  veb_node/.style = {align=center, inner sep=0pt, text centered, circle,
	  font=\sffamily, draw=black, text width=1.2em},
  block_l0/.style = {rectangle, draw=black, dashed, inner sep=7pt, draw=gray},
  block_l1/.style = {rectangle, draw=black, densely dotted, thin, inner sep=3pt, draw=gray},
  level/.style={level distance=1.7cm}
}
	%level/.style={sibling distance = 6cm/#1, level distance = 1.3cm}
\begin{tikzpicture}[
	scale=0.7,
	level 1/.style = {sibling distance=9.6cm},
	level 2/.style = {sibling distance=4.6cm},
	level 3/.style = {sibling distance=2.3cm},
	level 4/.style = {sibling distance=1cm},
]
	\node [veb_node] (Node0) {0}
	child{ node [veb_node] (Node1) {1}
		child { node [veb_node] (Node2) {2}
			child { node [veb_node] (Node4) {4}
				child { node [veb_node] (Node5) {5} }
				child { node [veb_node] (Node6) {6} }
			}
			child { node [veb_node] (Node7) {7}
				child { node [veb_node] (Node8) {8} }
				child { node [veb_node] (Node9) {9} }
			}
		}
		child { node [veb_node] (Node3) {3}
			child { node [veb_node] (Node10) {10}
				child { node [veb_node] (Node11) {11} }
				child { node [veb_node] (Node12) {12} }
			}
			child { node [veb_node] (Node13) {13}
				child { node [veb_node] (Node14) {14} }
				child { node [veb_node] (Node15) {15} }
			}
		}
	}
	child{ node [veb_node] (Node16) {16}
		child { node [veb_node] (Node17) {17}
			child { node [veb_node] (Node19) {19}
				child { node [veb_node] (Node20) {20} }
				child { node [veb_node] (Node21) {21} }
			}
			child { node [veb_node] (Node22) {22}
				child { node [veb_node] (Node23) {23} }
				child { node [veb_node] (Node24) {24} }
			}
		}
		child { node [veb_node] (Node18) {18}
			child { node [veb_node] (Node25) {25}
				child { node [veb_node] (Node26) {26} }
				child { node [veb_node] (Node27) {27} }
			}
			child { node [veb_node] (Node28) {28}
				child { node [veb_node] (Node29) {29} }
				child { node [veb_node] (Node30) {30} }
			}
		}
	};

	\node [block_l0,fit=(Node0)] (BlockT) {};
	\node [block_l0,fit=(Node1) (Node5) (Node15)] (BlockB0) {};
	\node [block_l0,fit=(Node16) (Node20) (Node30)] (BlockB1) {};

	\node [block_l1,fit=(Node1) (Node2) (Node3)] (BlockB0T) {};
	\node [block_l1,fit=(Node4) (Node5) (Node6)] (BlockB0B0) {};
	\node [block_l1,fit=(Node7) (Node8) (Node9)] (BlockB0B1) {};
	\node [block_l1,fit=(Node10) (Node11) (Node12)] (BlockB0B2) {};
	\node [block_l1,fit=(Node13) (Node14) (Node15)] (BlockB0B3) {};

	\node [block_l1,fit=(Node16) (Node17) (Node18)] (BlockB1T) {};
	\node [block_l1,fit=(Node19) (Node20) (Node21)] (BlockB1B0) {};
	\node [block_l1,fit=(Node22) (Node23) (Node24)] (BlockB1B1) {};
	\node [block_l1,fit=(Node25) (Node26) (Node27)] (BlockB1B2) {};
	\node [block_l1,fit=(Node28) (Node29) (Node30)] (BlockB1B3) {};
\end{tikzpicture}

\caption{The Van Emde Boas layout of a full binary tree of height 5.
Boxes mark recursive applications of the construction. The indices within
every box are contiguous.}
\label{fig:veb_layout_5}
\end{figure}

\begin{theorem}
In a tree stored in the Van Emde Boas layout, reading the sequence of nodes
on any root-leaf path costs $\Theta(1+\log_B N)$ memory transfers (assuming
every node fits in one block).
\end{theorem}

\begin{proof}
Let us examine the recursive applications of the Van Emde Boas construction
for a height $h=\log (N+1)$ and denote the heights of the bottom trees
$h_1, h_2, \ldots$. For example, as seen in Figure \ref{fig:veb_layout_5}, for
$h=5$ we have $h_1=4$, $h_2=2$ and $h_3=1$.

Let us assume that the entire tree doesn't fit in a block of size $B$.
Because a single node fits in one block, there exists a \textit{level of
detail} $i$ such that the $2^{h_i}-1$ nodes in the $i$-th-iteration bottom trees
take up less than $B$ nodes, but bottom trees of the $i-1$-th-iteration bottom
trees take up at least $B$ nodes.

Since $2^{h_i}-1 < B$ and $2^{h_i-1}-1 \geq B$, it follows that $h_i=\O(\log_B)$.
By construction, the path of from the root of the tree to any leaf contains
$\Theta(\log N/h_i)=\Theta(\log_B N)$ bottom trees from the $i$-th iteration.
The path also contains one top tree containing the root which may have a height
different than $h_i$, but the height of this tree is always $\leq h_i$.

Because the $2^{h_i}-1$ nodes of any $i$-th-iteration bottom tree are stored
in a contiguous array, every $i$-th-iteration bottom tree (or top tree)
can be read using $\O(1)$ block transfers. Since every root-to-leaf path
is composed of $\Theta(\log_B N)$ such subtrees, traversing the path
takes $\Theta(1+\log_B N)$ block transfers (the added $\O(1)$ covers
the $B\geq N$ case).
\end{proof}

The Van Emde Boas layout thus makes a fine data structure for querying static
data, but we need to combine it with another data structure to allow updates.

\subsection{Efficient implementation concerns}
A useful property of the Van Emde Boas layout is that it is fully specified
by the height of the tree, so storing a full binary tree in this layout does
not require keeping pointers -- the positions of left and right children
of a node can be calculated when they are needed. We refer to this property
of the layout as allowing \textit{implicit pointers}.

We will only use the Van Emde Boas order to walk from the root node to a leaf.
Given the \textit{Van Emde Boas ID} of a node,
we can easily calculate the Van Emde Boas IDs of its children by
a recursive procedure. This procedure either returns \textit{internal} node
pointers, referencing actual nodes of the tree, or it returns \textit{external}
indexes, which represent virtual nodes below the leaves, counted from left
to right.

TODO: figure?
TODO: check
\begin{algorithmic}
\Function {GetChildren} {$n$: node Van Emde Boas ID, $h$: tree height}
	\If {$h = 1$ and $n = 0$} \Return{external (0,1)} \EndIf

	\State {$h_\downarrow \gets \lhfloor h-1 \rhfloor$} \Comment{Calculate
top and bottom heights}
	\State {$h_\uparrow \gets h-h_\downarrow$} 
	\State {$N_\uparrow, N_\downarrow \gets 2^{h_\uparrow}-1,
	2^{h_\downarrow}-1$} \Comment{Calculate top and bottom tree sizes}

	\If {$n <  N_\uparrow$}
		\State $\ell, r \gets$ \Call{GetChildren}{$n$,$h_\uparrow$}
		\If {$\ell$ and $r$ are internal}
			\State \Return{internal ($\ell$,$r$)}
		\Else\Comment{$\ell$ and $r$ they point to bottom tree roots}
			\State \Return{internal
				($N_\uparrow+\ell\cdot N_\downarrow$,
				$N_\uparrow+r\cdot N_\downarrow$)
			}
		\EndIf
	\Else
		\State {$i \gets (n-N_\uparrow) \div N_\downarrow$}
			\Comment{The node $n$ lies within the $i$-th bottom tree.}
		\State {$b \gets N_\uparrow + i\cdot N_\downarrow$}
			\Comment{$b$ is the root of the $i$-th bottom tree.}
		\State $\ell,r\gets$ \Call{GetChildren}{$n-b$, $h_\downarrow$}
		\If {$\ell$ and $r$ are internal}
			\State \Return{internal ($\ell+b$, $r+b$)}
		\Else
			\State {$e \gets 2^{h_\downarrow}$} \Comment{Adjust
				indices by $e$ external nodes per bottom
				tree.}
			\State \Return{external ($\ell+i\cdot e$, $r+i\cdot e$)}
		\EndIf
	\EndIf
\EndFunction
\end{algorithmic}

The cost of this procedure in the cache-oblivious model is $\O(1)$, because
it can be implemented using constant memory (by modifying the recursion into
a loop). Unfortunately, on a real computer, this is not quite the case --
every call of this procedure performs $\Theta(\log\log N)$ instructions and
repeating this $\Theta(\log N)$ times between the root and a leaf is slow.
Indeed, this calculation of implicit pointers can become the performance
bottleneck of the cache-oblivious B-tree.
This can be slightly alleviated by caching the results for trees of small
height.

TODO: graf

TODO: efficient calculation of pointers

\section{Ordered file maintenance}
In this section we shall describe an auxilliary data structure which stores
an ordered list of $N$ keys (in the given order) in a contiguous array of
size $\O(N)$. The operations allowed on this structure are
\textsc{Insert}(\textit{index},\textit{key}), which inserts a new key before
or after a given index in the array, and \textsc{Delete}(\textit{index}),
which deletes an item at a given index.
We shall allow the array to contain gaps up to a constant maximum size.
Having constant-size gaps allows scanning a range of $n$ keys using the
optimal $\Theta(n/B)$ scans.

Updates (\textsc{Insert}s and \textsc{Delete}s) will rewrite a contiguous
range of size $\Theta(\log^2 N)$ amortized. Thanks to the range being
contiguous, updates will incur $\Theta(1+\frac{\log^2 N}{B})$ amortized
block transfers.

The data structure maintains certain density bounds on subranges of the array.
In particular, the array is divided into \textit{leaf blocks} of size $\O(\log N)$.
A \textit{virtual full binary tree} is built above those leaf blocks. Every
node of this binary tree represents the range covering all the leaf blocks below.

The densities in the leaf blocks are kept between \unichar{"00BC} % VULGAR FRACTION ONE QUARTER
and 1. When a block becomes too sparse or too dense, we walk up the binary
tree until we find a node that fits our density requirements.
Those requirements become stricter as we walk up the binary tree.
In particular, the density of a node of depth $d$ within a tree of height $h$
is kept between $\frac{1}{2}-\frac{1}{4}\frac{d}{h}$ and $\frac{3}{4}+\frac{1}{4}\frac{d}{h}$.
When we find a node that is \textit{within bounds}, we uniformly redistribute
the keys in its range.

We claim that while this while this redistribution may reorganize a range
of size up to $\O(N)$, the redistribution of a node puts all nodes below it
well within bounds, so the next redistribution of those nodes will occur
only after relatively many updates. % TODO: how much is "many"?

TODO: claim, proof

TODO: other approaches?
