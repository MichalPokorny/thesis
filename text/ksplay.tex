\chapter{$k$-splay trees}
\label{chapter:ksplay}
Splay trees are useful for exploiting the regularity of access patterns.
On the other hand, they do not make good use of memory blocks -- every node
is just binary, so walking to a~node in depth $d$ takes $\O(\log d)$ block
transfers. B-trees improve this by a~factor of $\log B$ by storing more keys
per node.

Self-adjusting $k$-ary search trees (or \emph{$k$-splay trees}) developed
by \cite{ksplay-sherk} use a~heuristic named \emph{$k$-splaying}
to dynamically adjust the structure of a~$k$-ary search tree according to
the query sequence. We have decided to implement and benchmark this data
structure because of the potential to combine self-adjustment for a~specific
search sequence with cache-friendly node sizes.

Similarly to B-trees, nodes in $k$-ary splay trees store up to $k-1$ key-value
pairs and $k$ child pointers. All nodes except the root are \emph{complete}
(i.e.\ they contain exactly $k-1$ pairs).

\mbox{B-tree} operations take $\Theta(\log_b N)$ node accesses. It can be
proven that \mbox{$k$-splay} tree operations need at most $\O(\log N)$
amortized node accesses, which matches ordinary splay trees, but unfortunately
$k$-splay trees do not match the optimal $\O(\log_k N)$ bound given by B-trees.
In particular, \cite{ksplay-nonopt} found a~family of counterexample trees,
in which repeatedly $k$-splaying deepest nodes requires $\log N$ node accesses.

$k$-splay trees are proven in \cite{ksplay-sherk} to be $(\log k)$-competitive
compared to static optimal $k$-ary search trees.

The $k$-splaying operation is conceptually similar to splaying in the original
splay trees of Sleator and Tarjan, but $k$-splaying for $k=2$ differs from
splaying in original splay trees. To disambiguate these terms, we will
reserve the term \emph{splaying} for Sleator and Tarjan's splay trees and
we will only use \emph{$k$-splaying} to denote the operation
on $k$-ary splay trees.

In our description of $k$-splaying, we shall begin with the assumption that
all nodes on the search path are complete, and we will later modify
the $k$-splaying algorithm to allow \emph{incomplete} and \emph{overfull}
nodes.
Because splay trees store one key per node, it is not necessary to distinguish
between splaying a~node and splaying a~key. In $k$-ary splay trees, we always
$k$-splay nodes.

As with Sleator and Tarjan's splaying, $k$-splaying is done in steps.

Splaying steps of Sleator and Tarjan's splay trees involve 3 nodes connected
by 2 edges, except for the final step, which may involve the root node and one
of its children connected by an edge. In every step, the splayed node (or key)
moves from the bottom of a~subtree to the root of a~subtree.
Splaying stops when the splayed node becomes the root of the splay tree.

In $k$-splay trees, $k$-splaying steps involve $k$ edges connecting
$k+1$ nodes, except for the final step which may involve fewer edges.
The objective of $k$-splaying is to move a~\emph{star node} to the
root of the tree. In every $k$-splaying step, we start with the star node
at the bottom of a~subtree, and we change the structure of the subtree
so that the star node becomes its root.
The initial star node is the last node reached on a~search operation, and
$k$-splaying will move the content of this node closer to the root.
However, the structural changes performed in $k$-splaying steps shuffle
the keys and pointers of involved nodes, so after a~$k$-splay step,
the star node may no longer contain the key we originally looked up.
Unlike splay trees, the distinction between $k$-splaying a~node and a~key
is therefore significant.

There are two types of $k$-splay steps: \emph{non-terminal steps} on
$k+1$ nodes, and \emph{terminal steps} on less than $k+1$ nodes.
We consider first the non-terminal steps.

Non-terminal $k$-splay steps examine the bottom $k+1$ nodes in the search
path to the current star node. Denote the nodes $p_0,\ldots p_{k}$, where $p_i$
is the parent of $p_{i+1}$ and $p_k$ is the current star node. A~non-terminal
step traverses these nodes in DFS order and collects \emph{external children}
(i.e.\ children that are not present in $\{p_0,\ldots p_k\}$) and all key-value
pairs of $p_0,\ldots p_k$ in sorted order. These children and key-value pairs
are \emph{fused} into a~``virtual node'' with $(k+1)\cdot (k-1)$ keys and
$(k+1)\cdot(k-1) + 1$ children. Finally, this ``virtual node'' is broken down
into a~fully balanced $k$-ary search tree of depth 2. There is exactly one
way to arrange this $k$-ary search tree, because all its nodes must be complete.
The root of this search tree becomes the new star node and a~new $k$-splay
step starts.

Terminal $k$-splay steps occur when we are $k$-splaying at most $k$
nodes, with the highest one always being the root of the $k$-splay tree.
Unlike non-terminal steps, terminal steps have too few nodes in the splayed
path, so they leave the root underfull. Otherwise, terminal steps are the same
as non-terminal steps.

\begin{figure}
\centering
\begin{tikzpicture}[]
\tikzstyle{ks_node}=[rectangle split, rectangle split horizontal,
	rectangle split ignore empty parts, draw]
\tikzstyle{ks_leaf}=[circle, draw, text depth=0em, text height=0.5em, scale=0.9]

\node at (-4,0) [ks_node] {15 \nodepart{two} 95} [->]
child { node [ks_leaf] {a} }
child { node [ks_node] {21 \nodepart{two} 56}
	child { node [ks_leaf] {b} }
	child { node [ks_leaf] {c} }
	child { node [ks_node] {71 \nodepart{two} 80}
		child { node [ks_node] (StarBefore) {60 \nodepart{two} 64}
			child { node [ks_leaf] {d} }
			child { node [ks_leaf] {e} }
			child { node [ks_leaf] {f} }
		}
		child { node [ks_leaf] {g} }
		child { node [ks_leaf] {h} }
	}
}
child { node [ks_leaf] {i} };
\node at ($(StarBefore.west)+(-0.3,0)$) {$\star$};

\node at (4,-1.5) [ks_node] (StarAfter) {56 \nodepart{two} 71} [->]
child[sibling distance=2.5cm] { node [ks_node] {15 \nodepart{two} 21}
	child[sibling distance=0.7cm] { node [ks_leaf] {a} }
	child[sibling distance=0.7cm] { node [ks_leaf] {b} }
	child[sibling distance=0.7cm] { node [ks_leaf] {c} }
}
child[sibling distance=2.5cm] { node [ks_node] {60 \nodepart{two} 64}
	child[sibling distance=0.7cm] { node [ks_leaf] {d} }
	child[sibling distance=0.7cm] { node [ks_leaf] {e} }
	child[sibling distance=0.7cm] { node [ks_leaf] {f} }
}
child[sibling distance=2.5cm] { node [ks_node] {80 \nodepart{two} 95}
	child[sibling distance=0.7cm] { node [ks_leaf] {g} }
	child[sibling distance=0.7cm] { node [ks_leaf] {h} }
	child[sibling distance=0.7cm] { node [ks_leaf] {i} }
};
\node at ($(StarAfter.west)+(-0.3,0)$) {$\star$};

\draw[->] (-1,-1.5) -- (1,-1.5);

\end{tikzpicture}
\caption{Non-terminal 3-splaying step with marked star nodes.}
\end{figure}

\textsc{Delete} and \textsc{Insert} operations employ variations
of the same procedure. We look up the node to update and we insert or
delete the key within the node. Then, we $k$-splay the node, while allowing
the star node to be temporarily overfull or underfull. After $k$-splaying
the updated node, the root becomes the star node. If it has only one
child, we merge it with the root, and if the root has more than $k$ children,
we split it.

By performing an analysis similar to splay trees, \cite{ksplay-sherk}
shows that the number of node reads performed in a~sequence of $m$
\textsc{Find}s in a~$k$-splay tree with $n$ keys is
$\O(m\log n+\frac{n}{k}\log n)$.
