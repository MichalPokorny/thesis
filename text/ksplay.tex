\chapter{$k$-splay trees}
Splay trees are useful for exploiting the regularity of access patterns.
On the other hand, they do not make good use of memory blocks -- every node
is just binary, so walking to a node in depth $d$ takes $\O(\log d)$ block
transfers. B-trees improve this by a factor of $\log B$ by storing more keys
per node.

Self-adjusting $k$-ary search trees (or \textit{$k$-splay trees}) developed
by Sherk\cite{ksplay-sherk} use a heuristic named \textit{$k$-splaying}
to dynamically adjust the structure of a $k$-ary search tree according to
the query sequence. We have decided to implement and benchmark this data
structure because of the potential to combine self-adjustment for a specific
search sequence with cache-friendly node sizes.

Similarly to B-trees, nodes in $k$-ary splay trees store up to $k-1$ key-value
pairs and $k$ child pointers. All nodes except the root are \textit{complete}
(i.e. contain exactly $k-1$ pairs).

B-tree operations take $\Theta(\log_b N)$ node accesses. It can be proven that
$k$-splay tree operations need at most $\O(\log N)$ amortized node accesses,
which matches ordinary splay trees, but unfortunately it is not known whether
does the optimal $\O(\log_k N)$ bound apply to $k$-splay trees.
% TODO: check?

The $k$-splaying operation is conceptually similar to splaying in the original
splay trees of Steator and Tarjan, but $k$-splaying for $k=2$ differs from
splaying in original splay trees. To disambiguate these terms, we will
reserve the term \textit{splaying} for Steator and Tarjan's splay trees and
we will only use \textit{$k$-splaying} to denote the operation
on $k$-ary splay trees.

In our description of $k$-splaying, we shall begin with the assumption that
all nodes on the search path are complete, and we will later modify
the $k$-splaying algorithm to allow \textit{incomplete} and \textit{overfull}
nodes.
Because splay trees store one key per node, it is not necessary to distinguish
between splaying a node and splaying a key. In $k$-ary splay trees, we always
$k$-splay nodes.

As with Steator and Tarjan's splaying, $k$-splaying is done in steps.
All splaying steps involve 3 nodes connected by 2 edges, except for the final
step, which may involve the root node and one of its children connected by
an edge. In every step, the splayed node (or key) moves from the bottom
of a subtree to the root of a subtree. Splaying stops when the splayed node
becomes the root of the splay tree.

In $k$-splay trees, $k$-splaying steps involve $k$ edges connecting
$k+1$ nodes, except for the final step which may involve fewer edges.
When $k$-splaying, the objective is to move a \textit{star node} to the
root of the tree. In every $k$-splaying step, we start with the star node
at the bottom of the subtree, and we change the structure of the $k+1$
nodes to put the star node to the root of the subtree.
The initial star node is the last node reached on a search operation, and
$k$-splaying will move the content of this node closer to the root.
The structural changes performed in $k$-splaying steps shuffle the keys and
pointers of involved nodes, which is why the distinction between splaying
a node and splaying a key is significant -- after a $k$-splay step,
the star node may no longer contain the key we were originally searching for.

There are two types of $k$-splay steps: \textit{non-terminal steps} on
$k+1$ nodes, and \textit{terminal steps} on less than $k+1$ nodes.
We consider first the non-terminal steps.

Non-terminal $k$-splay steps start a path of $k+1$ nodes $p_0,\ldots p_{k}$,
where $p_i$ is the parent of $p_{i+1}$. A non-terminal step traverses these
nodes in DFS order and collects \textit{external children} (i.e. children
that are not present in $\{p_0,\ldots p_k\}$) and all key-value pairs of
$p_0,\ldots p_k$ in sorted order.
These children and key-value pairs are \textit{fused} into a "virtual node"
with $(k+1)\cdot (k-1)$ keys and $(k+1)\cdot(k-1) + 1$ children. Finally,
this "virtual node" is broken down into a fully balanced $k$-ary
search tree of depth 2. There is only one unique way to arrange this
$k$-ary search tree, because all its nodes must be complete. The root of this
search tree becomes the new star node and a new $k$-splay step starts.

TODO: terminal case
TODO: amortized access time
TODO: insert
TODO: delete
