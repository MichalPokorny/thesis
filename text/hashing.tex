\chapter{Hash tables}
\label{chapter:hashing}
When implementing an unordered dictionary, hashing is the most common tool
\footnote{Indeed, one of the many established meanings of the word ``hash''
	is ``unordered dictionary''.}
Hashing is a very old idea and the approaches are numerous. Hashing
techniques usually allow amortized constant-time \textsc{Find}, \textsc{Insert}
and \textsc{Delete} at the expense of disallowing \textsc{FindNext} and
\textsc{FindPrevious}. Certain schemes provide stronger than expected-time
bounds, like deterministic constant time for \textsc{Find}s in cuckoo hashing
(described in section \ref{sec:cuckoo}).

The idea of hashing is reducing the size of the large key universe $U$ to
a smaller \emph{hash space} $H$ via a \emph{hashing function}
$h\mathop{:}U\rightarrow H$.
Let us denote the size of the hash space $M$.
For a key $k$, we call $h(k)$ the \emph{hash} of $k$.

The size of the hash space is selected small enough to allow using hashes
of keys as indices in an array, which we call the \emph{hash table}.
The $i$-th element of the hash table may either be empty, or it may contain
a key-value pair $(k,v)$, where $h(k)=i$.

TODO: figure

As long as all inserted keys have distinct hashes, hash tables are easy:
\textsc{Find}s, \textsc{Insert}s and \textsc{Delete}s all consist of just
hashing the key and performing one operation in the hash table -- it takes just
constant time, so also a constant number of memory transfers.
Unfortunately, if the set of stored keys is not known in advance,
some keys $k_1\neq k_2$ may have the same hash value.
This condition is called a \emph{collision}.

Specific collision resolution strategies differ between hashing approaches.

\section{Separate chaining}
\emph{Separate chaining} stores the set of all key-value pairs with
$h(k)=i$ in slot $i$. Let us call this set the \emph{collision chain} for hash
$i$, and denote its size $C_i$. The easiest solution is storing a pointer to
a linked list of colliding key-value pairs in the hash table. If there are no
collisions in an occupied slot, an easy and common optimization is storing
the only key-value pair directly in the hash table.

TODO: not very cache-friendly for long chains

TODO: uniform ::= pravdepodobnost = O(1/m)

If we use separate chaining, the costs of all hash table operations become
$\O(1)$ for slot lookup and $\O(|C_i|)$ for scanning the collision chain.
By picking a hash function that evenly distributes hashes among keys,
we can prove that the expected length of a collision chain is short.

To keep the expected chain length low, every time the hash table increases
or decreases in size by a constant factor, we rebuild it with a new $M$
picked to be $\O(N)$. The rebuild time is $\O(1)$ amortized per operation.

If we pick the hash function $h$ at random (i.e. by independently randomly
assigning $h(k)$ for all $k\in U$), we have $\forall i\in H, k\in U:
\Pr[h(k)=i]=\frac{1}{M}$, so $\E[C_i]=\frac{N}{M}$, which is constant if we
maintain $M=\O(N)$, so the expected time per operation is also constant.
The $\frac{N}{M}$ ratio is commonly called the \emph{load factor}.

However, storing a random hash function would require $|U| \log M$ bits, which
is too much if we only store a few keys from a very large universe. In practice,
we pick a hash function from a certain smaller family according to a formula
with some variables chosen at random.
Given a family $\mathcal{H}$ of hashing functions, we call $\mathcal{H}$
\emph{universal} if $\Pr_{h\in \mathcal{H}}[h(x)=h(y)]=\O(\frac{1}{M})$ for any
$x, y\in U$, and \emph{strongly universal} if this probability is
$\leq\frac{1}{M}$. % TODO: Do I need strong universality?
For any universal family of hash functions, the expected time per operation
is constant when using chaining.

A family $\mathcal{H}$ is $t$-independent if the hashes of $t$ distinct keys
$k_1,\ldots k_t$ are ``asymptotically independent'':
$$\forall k_1\neq\ldots \neq k_t\subseteq K,
	h_1\ldots h_t\subseteq H: \Pr_{h\in H}[h(k_1)=h_1\wedge \ldots \wedge
	h(k_t)=h_t]=\O(m^{-t})$$
2-independent hash function families are universal, but universal families
are not necessarilly 2-independent.
Trivially, random hash functions are $k$-wise independent for any $k\leq |U|$.

with cache of Omega(log n) totally random -> O(1) amortized per operation w.h.p.

%TODO: variance: constant
%interesting: E[C_t^2], depends on $h$
%$E[C_t^2]=1/m \sum_s E[C_s^2]$ = 1/m (pocet kolizi) = 1/m \sum
%Pr_{i,j}{h(x_j)=h(y_j)}=O(1) by univ.

%w.h.p.: 1-1/n^c, c arb.

"pretty rare in practice", because hashing twice

%"linear probing: basically free" (10\% more than memory access)
%cuckoo: parallelism in computing hashes

Some common universal families of hash functions include:
\begin{itemize}
\item $h(k)=((a\cdot k)\bmod p)\bmod M$, where $p$ is a prime $\geq |U|$
	and $a\in\{0,\ldots p-1\}$ (\cite{univ-classes}).
	% TODO: not 2-independent
	The collision probability for this hash function is
	$\frac{\lfloor p/M\rfloor}{p-1}$, so it becomes less efficient
	if $M$ is close to $p$.
\item $(a\cdot k)\gg(\log u-\log m)$ for $M, |U|$ powers of 2
	(\cite{dietzfelbinger}).
	This hash function replaces modular arithmetic by bitwise shifts,
	which are faster on hardware.
\item \emph{Simple tabulation hashing} (\cite{univ-classes}):
	% TODO: check citation
	interpret the key $x\in K$ as a vector
	of $c$ equal-size components $x_1,\ldots x_c$. Pick $c$ random hash
	functions $h_1,\ldots h_c$ mapping from key components $x_i$ to $H$.
	The hash of $x$ is computed by taking the bitwise XOR ($\oplus$)
	of hashes of all components $x_i$:
	$$h(x)=h_1(x_1)\oplus h_2(x_2)\oplus \ldots \oplus h_c(x_c)$$
	Simple tabulation hashes take time $\O(c)$ to compute in the general RAM
	model and it needs space $\O(c\cdot |U|^{1/c})$.  % TODO: approx u^\epsilon

	Simple tabulation hashing is 3-independent and not 4-independent,
	but a recent analysis in \cite{power-of-simple-tab} showed that
	it provides surprisingly good properties in some applications,
	which we will mention later.
\end{itemize}

While random hash functions combined with chaining give an expected $\O(1)$
time per operation, with high probability (i.e.  $P=1-N^{-c}$ for an arbitrary
choice of $c$) there is at least one chain with at least $\O(\log N/\log\log N)$
items. The high-probability worst-case bound on operation time also applies
to hash functions with a high independence ($\O(\log N/\log\log N)$)
(\cite{chernoff-hoeffding-bounds}) and simple tabulation
hashing (\cite{power-of-simple-tab}).

\section{Perfect hashing}
\emph{Perfect hashing} avoids the issue of collisions by picking a hashing
function that is collision-free for the given key set. If we fix $K$
in advance (i.e.\ if we perform \emph{static hashing}), we can use a variety
of algorithms which produce a collision-free hashing function at the cost
of some preprocessing time. If the hash function produces no empty slots
(i.e.\ $M=N$), we call it \emph{minimal}.

For example, HDC (\textit{Hash, Displace and Compress},
\cite{hdc-hashing}) is a randomized algorithm that can generate a perfect
hash function in expected $\O(N)$ time. The hash function can be represented
using 1.98 bits per key for $M=1.01 N$, or more efficiently if we allow a larger
$M$. All hash functions generated by HDC can be evaluated in constant time.
The algorithm can be simply generalized to build $k$-perfect hash functions,
which allow up to $k$ collisions per slot. The \textit{C Minimum Perfect
Hashing Library}, available at \url{http://cmph.sourceforge.net/}, implements
HDC along with several other minimum perfect hashing algorithms.

A dynamic version of perfect hashing, commonly referred to as \emph{FKS
hashing} after the initials of the authors, was developed in \cite{fks-hashing}.
FKS hashing takes worst-case $\O(1)$ time for queries (an improvement over
expected $\O(1)$ with chaining) and expected amortized $\O(1)$ for updates.

FKS hashing is two-level. The first-level hashing function $f$ partitions
$K$ into $M$ buckets $B_1,\ldots B_M$. Denote their sizes as $b_i$.
Every bucket is stored in a separate hash table, mapping $B_i$ to an array
of size $\Theta(b_i^2)=\beta b_i^2$ via its private hash function $g_i$.
The constant $\beta$ will be picked later.
Each function $g_i$ is injective: buckets may contain no collisions.
% TODO: how can we store the fully random HF? why does it matter? maps from B_i?

If we pick the first-level hash function $f$ from a universal family
$\mathcal{F}$, the expected total size of all buckets is linear, so an FKS
hash table takes expected linear space:
$$\E\left[\sum_{i=1}^N b_i^2\right]=
	\sum_{i=1}^N \sum_{j\in\{1,\ldots N\}\smallsetminus\{i\}}
	\Pr_{f\in\mathcal{F}}[f(k_i)=f(k_j)]=
	\O\left(N^2\cdot\frac{1}{M}\right)=\O(N)$$
We pick $f$ at random from $\mathcal{F}$ until we find one that will need at
most $\sum_{i=1}^N b_i=\alpha N$ space, where $\alpha$ is an appropriate
constant. Picking a proper $\alpha$ yields expected $\O(N)$ time to pick $f$.
TODO

To select a second-level hash function $g_i$, we pick one randomly from
a universal family $\mathcal{G}$ until we find one that gives no collisions
in $B_i$. By universality of $g_i$, the expected number of collisions in $B_i$
is constant:
$$\E_{g_i\in\mathcal{G}}[\text{\# of collisions in }B_i]=
	{b_i\choose 2}\cdot\O\left(\frac{1}{b_i^2}\right)=\O(1)$$
By tuning the constant $\beta$, we can make the expected number of collisions
small (e.g. $\leq\frac{1}{2}$), so we can push the probability of having no
collisions above a constant (e.g. $\geq\frac{1}{2}$). This ensures that for
every bucket, we will find an injective hash function in expected $\O(1)$
trials.

To \textsc{Find} a key $k$, we simply compute $f(k)$ to get the right bucket,
and we look at position $g_{f(k)}(k)$ in the bucket, which takes deterministic
$\O(1)$ time.

FKS was extended to allow updates by \cite{dyn-ph-bounds}.
We maintain $\O(N)$ buckets, and whenever $N$ increases or decreases by
a constant factor, we rebuild the whole FKS hash table in $\O(N)$, which
amortizes to $\O(1)$ per operation. Each bucket $B_i$ has $\O(b_i^2)$ slots.
Whenever $b_i$ increases or decreases by a constant factor (e.g. 2), we resize
the reservation by the constant factor's square (e.g. 4).
The expected amortized time for \textsc{Insert} and \textsc{Delete} is $\O(1)$.
\cite{univ-class-of-hfns} enhances this to $\O(1)$ with high probability.

\section{Open addressing}
When we attempt to \textsc{Insert} a new pair with key $k$ into an occupied slot
$h(k)$, we can start trying out alternate slots $a(k,1), a(k,2), \ldots$
until we succeed in finding an empty slot. We call this approach \emph{open
addressing}. Examples of choices of $a(k,x)$ include:
\begin{itemize}
\item \emph{Linear probing}: $a(k,x)=(h(k)+x) \bmod M$
\item \emph{Quadratic probing}: $a(k,x)=(h(k)+x^2) \bmod M$
\item \emph{Double hashing}: $a(k,x)=[h(k)+x\cdot (1+h'(k))]\bmod M$, where
$h'$ is a secondary hash function
\end{itemize}

When using this family of strategies, one also needs to slightly change
\textsc{Find} and \textsc{Delete}: \textsc{Find} must properly traverse
all possible locations that may contain the sought key, and \textsc{Delete}
and \textsc{Insert} must ensure that \textsc{Find} will know when to abort
the search.

To illustrate this point, consider linear hashing with $h(A)=h(C)=1$ and
$h(B)=2$. After inserting $A$ and $B$, slots 1 and 2 are occupied.
Inserting $C$ will skip slots 1 and 2, and $C$ will be inserted into slot 3.
When we try to look for $C$ later, we need to know that there are exactly 2 keys
that hash to 1, so we won't abort the search prematurely after only seeing
$A$ and $B$.

The ends of collision chains can be marked for example by explicitly maintaining
the lengths of collision chains in an array, or by marking the ends of chains
with a bit flag.

This requires all \textsc{Insert}s and \textsc{Delete}s to traverse an entire
collision chain to shorten it. The cost of \textsc{Delete}s can be reduced to
just the length of the collision chain by \emph{lazy deletion}. Deleted elements
are not replaced by elements from the end of the chain, but they are instead
just marked as deleted. \textsc{Find} then skips over deleted elements and
\textsc{Insert}s are allowed to overwrite them.
\textsc{Find}s also ``execute'' the deletions by overwriting the first
encountered deleted element by an element from the right. An analysis
of lazy deletions is presented in \cite{lazy-deletions}.

% TODO: document how exactly in our implementation
The reason why some implementations use quadratic probing and double hashing
over linear probing is that linear probing creates long chains when space is
tight. A chain covering hash values $[i;j]$ forces any key hashing to this
interval to take $\O(j-i)$ time per operation and to extend the chain further.

However, linear probing performs well if we can avoid long chains: it has
much better locality of reference.
\cite{knuth-linear} showed that if $M=(1+\varepsilon) N$, then using a random
hash function gives expected time $\O(1/\varepsilon^2)$ with linear probing.
\cite{linear-probing-ci} proved that 5-independent hash functions suffice
to get this bound $\O(1)$, and \cite{linear-probing-constant}
showed that the bound is tight (4-independence does not suffice).
\cite{power-of-simple-tab} gives a proof that simple tabulation hashing,
which is only 3-independent and which can be implemented faster than usual
5-independent schemes, also archieves $\O(1/\varepsilon^2)$.

\section{Cuckoo hashing}
\label{sec:cuckoo}

Cuckoo hashing was invented by \cite{cuckoo-hashing}.
A cuckoo hash table is composed of two hash tables $L$ and $R$ of equal size
$M=(1+\varepsilon)\cdot N$.
Two separate hashing functions $h_l$ and $h_r$ are associated with $L$ and $R$.
All key-value pairs $(k,v)$ are stored either in $L$ at position $h_l(k)$, or
in $R$ at position $h_r(k)$.

The cuckoo hash table can also be visualized as a bipartite ``cuckoo graph'',
where partities are slots in $L$ and $R$. Edges correspond to stored keys:
a stored key $k$ connects $h_l(k)$ in $L$ and $h_r(k)$ in $R$. The assignment
of key-value pairs to slots in $L$ or $R$ forms a matching on the graph.

\textsc{Find}s take $\O(1)$ worst-case time: they compute $h_l(k)$ and $h_r(k)$
and look at the two indices in $L$ and $R$. The two reads are independent and
can potentially be executed in parallel (TODO: superscalar CPU).

The load factor of the cuckoo hash table needs to be kept between two
constants, so \textsc{Delete} takes $\O(1)$ amortized time.

To \textsc{Insert} a new pair $(k,v)$, we examine slots $L[h_l(k)]$ and
$R[h_r(k)]$. If either slot is empty, $(k,v)$ is inserted there. Otherwise,
\textsc{Insert} tries to free up one of the slots, for example $L[h_l(k)]$.
If $L[h_l(k)]$ currently stores the key-value pair $(k_1,v_1)$, we attempt to
move it to $R[h_r(k_1)]$. If $R[h_r(k_1)]$ is occupied, we continue following
the path $k,k_1,\ldots$ until we either find an empty slot, or we find a cycle.
The name ``cuckoo hashing'' refers to this procedure by analogy with
cuckoo chicks pushing eggs out of host birds' nests.

TODO: figure

If we find a cycle both on the path from $L[h_l(k)]$ and on the path from
$R[h_r(k)]$ (we can follow both at the same time), the cuckoo graph with
the added edge $(h_l(k),h_r(k))$ has no matching. To assign key-value pairs
to slots, we pick new hashing functions $h_l$ and $h_r$ and we rebuild
the entire structure.

TODO: what if we use more hash tables?

TODO: "More Robust Hashing: Cuckoo Hashing with a Stash"

Cuckoo hashing requires better randomness properties to work efficiently.
On a $\Theta(\log N)$-independent or random hash function, updates take expected
time $\O(1)$ and the failure probability (i.e.\ probability that a cycle will
force us to rebuild the entire structure) is $\O(\frac{1}{N})$
(\cite{cuckoo-hashing}).
Hash functions with at most 6-independence are not enough -- their failure
probability of $\O(1-\frac{1}{N})$ is very bad
(\cite{cuckoo-hashing-indep-bounds}).
The build failure probability when using simple tabulation hashing is $\O(N^{1/3})$
(\cite{power-of-simple-tab}).

We will use a $\Theta(\log N)$-independent hashing scheme to get a low failure
probability. One simple $k$-independent family is $h(x)=(\sum_{i=0}^{k-1} a_i x^i
\bmod p) \bmod M$, where $p$ is a prime greater than $|U|$ and all $a_i$ are
chosen at random from $\Z_p$ (\cite{new-hash-fns}).
This hash function family is considered slow due to needing $\Theta(\log N)$
time and using integer division by $p$\footnote{
	% TODO: This footnote looks like "p^2".
	As \cite{univ-classes} points out, using Mersenne primes
	(i.e.\ $2^i-1$ for some $i$) enables a faster implementation of the
	modulo operation. On 64-bit keys, we could use $p=2^{89}-1$.
	It might not be possible to generalize this approach, because
	the question whether there are infinitely many Mersenne primes remains
	open.
}.

Other hashing schemes that obtain the needed $\Theta(\log N)$ independence
are presented in \cite{tab-based-4uni-hashing} and \cite{siegel-1995}.
The first scheme provides $k$-independent hash function for a general $k$,
with evaluation times dependent on $k$.
% TODO: f(k) je velmi divne
In order to outperform B-trees, we would like to use a hashing function
which evaluates faster than $\log N$. The latter scheme takes only $\O(1)$ time
per query if $k=\Theta(\log N)$. Unfortunately, constructing a hash function
by Siegel's scheme has high constant factors, so it is not very practical
(\cite{pagh-phd}). Both schemes need $\O(N^\epsilon)$ space
to store the function.

Drawback: large precomputed tables. CW-trick just requires $a_0,\ldots a_k-1$

$k$-universal hashing, $q$ characters $x_0,\ldots x_{1-1}, x_i\in [2^c]$

1) Construct $q+r$ \emph{derived characters} $z_0,\ldots z_{q+r-1}, z_j\in[p],
p\geq\max\{2^c,q+r\}$ (some of derived characters may be input characters)

2) $q+r$ independent tabulated hash functions $h_j$ into $[2^l]$,
$h(x)=\bigoplus h_j[z_j]$
