\chapter{Splay trees}
\label{chapter:splay}
Splay trees are a well-known variant of binary search trees introduced
by \cite{splay} that adjust their structure with every operation, including
lookups. Splay tree operations take $\O(\log N)$ amortized time, which is
similar to common balanced trees, like AVL trees or red-black trees.

However, unlike BSTs, splay trees adjust their structure
according to the access pattern, and a family of theorems proves that
various non-random access patterns are fast on splay trees.

Let us also denote stored keys as $K_1,\ldots K_N$ in sorted order.
Splay trees are binary search trees, so they are composed of nodes, each of
which contains one key and its associated value. A node has up to two children,
denoted \emph{left} and \emph{right}.
The left subtree of a node with key $K_i$ contains only keys smaller than $K_i$
and, symetrically, the right subtree contains only keys larger than $K_i$.

Splay trees are maintained via a heuristic called \emph{splaying}, which
moves a~specified node~$x$ to the root of the tree by performing a~sequence
of edge rotations along the path from the root to $x$.
Rotations cost $\O(1)$ time each and rotating any edge maintains
the soundness of search trees. Rotations are defined by
Figure~\ref{fig:rotation}.

\newcommand{\hunk}[2]{ node [splay_hunk] (hunk#1-#2) {#1} }

\begin{figure}
\centering
% TODO: DRY
\begin{tikzpicture}[
	splay_node/.style = {align=center, inner sep=0pt, text centered, circle,
	font=\sffamily, draw=black, text width=1.2em},
	triangle/.style = {regular polygon, regular polygon sides=3, scale=0.8, fill=white},
	splay_hunk/.style = {align=center, inner sep=0pt, text centered,
		triangle, font=\sffamily, draw=black, text width=1.2em}
]
\node [splay_node] at (0, 0) (x-b) {$x$}
child { node [splay_node] (y-b) {$y$}
	child { \hunk{1}{b} }
	child { \hunk{2}{b} }
}
child { \hunk{3}{b} };

\node [fit=(x-b) (y-b) (hunk1-b) (hunk2-b) (hunk3-b)] (Before) {};

\node [splay_node] at (6, 0) (y-a) {$y$}
child { \hunk{1}{a} }
child { node [splay_node] (x-a) {$x$}
	child { \hunk{2}{a} }
	child { \hunk{3}{a} }
};

\node [fit=(x-a) (y-a) (hunk1-a) (hunk2-a) (hunk3-a)] (After) {};
\path[->] ($(Before.east) + (+.5,+.5)$) edge node [above] {rotate right} ($(After.west)
+ (-.5,+.5)$);
\path[->] ($(After.west) - (+.5,+.5)$) edge node [below] {rotate left} ($(Before.east)
- (-.5,+.5)$);
\end{tikzpicture}
\caption{Left and right rotation of the edge between $x$ and $y$.}
\label{fig:rotation}
\end{figure}

To splay a node $x$, we perform \emph{splaying steps}, which
rotate edges between $x$, its parent $p$ and possibly its grandparent $g$
until $x$ becomes the root. Splaying a~node of depth $d$ takes time $\Theta(d)$.
Up to left-right symmetry, there are three cases of splaying steps:

\begin{description}
\item[Case 1 (\emph{zig}):] If $p$ is the root, rotate the $px$ edge. (This case is terminal.)
\item[Case 2 (\emph{zig-zig}):] If $p$ is not the root and $x$ and
		$p$ are both left or both right children, rotate
		the edge $gp$, then rotate the edge $px$.
\item[Case 3 (\emph{zig-zag}):] If $p$ is not the root and $x$ is a left child and $p$
		is a right child (or vice versa), rotate the edge $px$ and
		then rotate the edge now joining $g$ with $x$.
\end{description}

\begin{figure}
\centering
% TODO: DRY
\begin{tikzpicture}[
	splay_node/.style = {align=center, inner sep=0pt, text centered, circle,
		font=\sffamily, draw=black, text width=1.2em},
	triangle/.style = {regular polygon, regular polygon sides=3, scale=0.8,
		fill=white},
	splay_hunk/.style = {align=center, inner sep=0pt, text centered,
		triangle, font=\sffamily, draw=black, text width=1.2em},
]
\node at (0,0) (ZigBefore) {
	\begin{tikzpicture}
	\node [splay_node] at (0,0) (p-b) {$p$}
	child { node [splay_node] (x-b) {$x$}
		child { \hunk{1}{b} }
		child { \hunk{2}{b} }
	}
	child { \hunk{3}{b} };
	\end{tikzpicture}
};
\node at (6,0) (ZigAfter) {
	\begin{tikzpicture}
	\node [splay_node] at (0,0) (x-a) {$x$}
	child { \hunk{1}{a} }
	child { node [splay_node] (p-a) {$p$}
		child { \hunk{2}{a} }
		child { \hunk{3}{a} }
	};
	\end{tikzpicture}
};
\path[->] (2.5,0) edge node [above] {zig} (3.5,0);
\node at (0,-5) (ZigZigBefore) {
	\begin{tikzpicture}[]
	\node [splay_node] at (0,0) (g-b) {$g$}
	child { node [splay_node] (p-b) {$p$}
		child { node [splay_node] (x-b) {$x$}
			child { \hunk{1}{b} }
			child { \hunk{2}{b} }
		}
		child { \hunk{3}{b} }
	}
	child { \hunk{4}{b} };
	\end{tikzpicture}
};
\node at (6,-5) (ZigZigAfter) {
	\begin{tikzpicture}[]
	\node [splay_node] at (0,0) (x-a) {$x$}
	child { \hunk{1}{a} }
	child { node [splay_node] (p-a) {$p$}
		child { \hunk{2}{a} }
		child { node [splay_node] (g-a) {$g$}
			child { \hunk{3}{a} }
			child { \hunk{4}{a} }
		}
	};
	\end{tikzpicture}
};
\path[->] (2.5,-5) edge node [above] {zig-zig} (3.5,-5);
\node at (0,-11) (ZigZagBefore) {
	\begin{tikzpicture}[]
	\node [splay_node] at (0,0) (g-b) {$g$}
	child { node [splay_hunk] (hunk1-b) {1} }
	child { node [splay_node] (p-b) {$p$}
		child { node [splay_node] (x-b) {$x$}
			child { \hunk{2}{b} }
			child { \hunk{3}{b} }
		}
		child { \hunk{4}{b} }
	};
	\end{tikzpicture}
};
\node at (6,-11) (ZigZagAfter) {
	\begin{tikzpicture}[]
	%\node [splay_node] at (6,-0.75) (x-a) {$x$}
	\node [splay_node] at (0,0) (x-a) {$x$}
	[sibling distance=2.5cm]
	child { node [splay_node] (g-a) {$g$}
		[sibling distance=1.2cm]
		child { \hunk{1}{a} }
		child { \hunk{2}{a} }
	}
	child { node [splay_node] (p-a) {$p$}
		[sibling distance=1.2cm]
		child { \hunk{3}{a} }
		child { \hunk{4}{a} }
	};
	\end{tikzpicture}
};
\path[->] (2.5,-11) edge node [above] {zig-zag} (3.5,-11);
\end{tikzpicture}

\caption{The cases of splay steps.}
\label{fig:splay-step}
\end{figure}

% Splaying reduces the depth of every node along the access path by roughly
% one half. TODO: fakt? TODO: figure

To \textsc{Find} a key $K_F$ in a splay tree, we use binary search as in any
binary search tree: in every node, we compare its key $K_i$ with $K_F$, and
if $K_i \neq K_F$, we continue to the left subtree or to the right subtree.
If the subtree we would like to descend into is empty, the search is aborted,
since $K_F$ is not present in the tree.
After the search finishes (successfully or unsuccessfully), we splay the last
visited node. An \textsc{Insert} is similar: we find the right place for the
new node, insert it, and finally splay it up.

\textsc{Delete}s are slightly more complex. We first find and splay the node $x$
that we need to delete. If $x$ has one child after splaying, we delete it and
replace it by its only child. One possible way to handle the case of two
children is finding the rightmost node $x^-$ in $x$'s left subtree and splaying
it just below $x$. By definition, $x^-$ is the largest in $x$'s left subtree,
so it must have no right child after splaying below $x$. We delete $x$ and link
its right subtree below $x^-$.

\section{Time complexity bounds}
In this section, we follow the analysis of splay tree performance from
\cite{splay}.

For the purposes of analysis, we assign a~fixed positive weight $w(K_i)$
to every key~$K_i$. Setting $w(K_i)$ to $1/N$ allows us to prove that
the amortized complexity of the splay operation is $\O(\log N)$, which implies
that splay tree \textsc{Find}s and updates also take amortized logarithmic time.
Proofs of further complexity results will use a similar analysis with
other choices of the key-weight assignment function~$w$.

Let us denote the subtree rooted at a node $x$ as $T_x$
and the key stored in node $x$ as $T[x]$. Define the size $s(x)$ of a node $x$
as $\sum_{y\in T_x} w(T[y])$ and the rank $r(x)$ as $\log s(x)$.
For amortization, let us define the potential $\Phi$ of the tree to be the sum
of the ranks of all its nodes. In a sequence of $M$ accesses, if the $i$-th
access takes time $t_i$, define its amortized time $a_i$ to be
$t_i+\Phi_{i-1}-\Phi_{i}$. Expanding and telescoping the total access time
yields $\sum_{i=1}^M t_i=\Phi_0-\Phi_M+\sum_{i=1}^m a_i$.

We will bound the magnitude of every $a_i$ and of the total potential drop
$\Phi_0-\Phi_M$ to obtain an upper bound on $\sum_{i=1}^M t_i$.
Since every rotation can be performed in $\O(1)$
pointer assignments, we can measure the time to splay a node in the number
of rotations performed (or 1 if no rotations are needed).

\begin{lemma}
The amortized time $a$ to splay a node $x$ in a tree with root $t$
is at most $3(r(t)-r(x))+1 = \O(\log(s(t)/s(x)))$.
\end{lemma}
\begin{proof}
If $x=t$, we need no rotations and the cost of splaying is 1.
Assume that  $x\neq t$, so at least one rotation is needed.
Consider the $j$-th splaying step.
Let~$s$~and~$s'$, $r$~and~$r'$ denote the size and rank functions just before
and just after the step, respectively. We show that the amortized time
$a_j$ for the step is at most $3(r'(x)-r(x))+1$ in case 1
and at most $3(r'(x)-r(x))$ in case 2 or case 3. Let $p$ be the original parent
of $x$ and $g$ the original grandparent of $x$ (if it exists).

The upper bound on $a_j$ is proved by examining the structural changes made
by each case of a splay step, as shown on Figure \ref{fig:splay-step}.

\begin{description}
\item[Case 1 (\emph{zig}):]
	The only needed rotation of the $px$ edge may only change the rank
	of~$x$ and~$p$, so $a_j=1+r'(x)+r'(p)-r(x)-r(p)$.
	Because $r(p)\geq r'(p)$, we also have $a_j\leq 1+r'(x)-r(x)$.
	Furthermore, $r'(x)\geq r(x)$, so $a_j\leq 1+3(r'(x)-r(x))$.

\item[Case 2 (\emph{zig-zig}):]
	The two rotations may change the rank of $x$, $p$ and $g$,
	so $a_j=2+r'(x)+r'(y)+r'(z)-r(x)-r(y)-r(z)$. The zig-zig step
	moves $x$ to the original position of $g$, so $r'(x)=r(g)$.
	Before the step, $x$ was $p$'s child, and after the step,
	$p$ was $x$'s child, so $r(x)\leq r(p)$ and $r'(p)\leq r'(x)$.
	Thus $a_j\leq 2+r'(x)+r'(g)-2r(x)$.
	We claim that this is at most $3(r'(x)-r(x))$, that is, that
	$2r'(x)-r(x)-r(g)\geq 2$.

	$2r'(x)-r(x)-r'(g)=-\log\frac{s(x)}{s'(x)}-\log\frac{s'(g)}{s'(x)}$.
	We have $s(x)+s'(g)\leq s'(x)$.

	If $p,q\geq 0$ and $p+q\leq 1$, $\log p+\log q$ is maximized
	by setting $p=q=\frac{1}{2}$, which yields $\log p+\log q=-2$.
	Thus $2r'(x)-r(x)-r(g)\geq 2$ and $a_j \leq 3(r'(x)-r(x))$.

\item[Case 3 (\emph{zig-zag}):]
	The amortized time of the zig-zag step is
	$a_j=2+r'(x)+r'(p)+r'(g)-r(x)-r(p)-r(g)$. $r'(x)=r(g)$ and $r(x)\leq r(p)$,
	so $a_j\leq 2+r'(p)+r'(g)-2r(x)$.
	We claim that this is at most $2(r'(x)-r(x))$,
	i.e.,\ $2r'(x)-r'(p)-r'(g)\geq 2$. This can be proven
	similarly to case 2 from the inequality $s'(p)+s'(g)\leq s'(x)$.
	Thus we have $a_j \leq 2(r'(x)-r(x))\leq 3(r'(x)-r(x))$.
\end{description}

Telescoping $\sum a_j$ along the splayed path yields an upper bound of
$a \leq 3(r'(t)-r(x))+1$ for the entire splaying operation.
\end{proof}

Note that over any sequence of splay operations
% TODO: key vs. node assignment: w(i)
the potential $\Phi$ may only drop by up to $\sum_{i=1}^N \log(W/w(K_i))$
where $W=\sum_{i=1}^N w(K_i)$, since the size of the node containing $K_i$
must be between $w(K_i)$ and $W$.

\begin{theorem}[Balance Theorem]
A sequence of $M$ splay operations on a splay tree with $N$ nodes takes time
$\O((M+N)\log N+M)$.
\end{theorem}
\begin{proof}
Assign $w(K_i)=1/N$ to every key $K_i$. The amortized
access time for any key is at most $3\log N+1$. We have $W=1$, so
the potential drops by at most $\sum_{i=1}^N \log(1/w(K_i))=
\sum_{i=1}^N \log N=N\log N$ over the access sequence. Thus the time to perform
the access sequence is at most $(3\log N+1)M + N\log N = \O((M+N)\log N + M)$.
\end{proof}

The balance theorem states that, in the amortized sense, lookups in a~splay
tree are as efficient as in usual balanced search trees (like AVL or red-black
trees).

\vskip 1em

Splay trees provide stronger time complexity guarantees than just amortized
logarithmic time per access.
For any key $K_i$, let $q(K_i)$ be the number of occurences of $K_i$
in an access sequence.
\begin{theorem}[Static Optimality Theorem]
If every key is accessed at least once, the total access time is
$\O(M + \sum_{i=1}^N q(K_i)\log\frac{M}{q(K_i)})$.
\end{theorem}
\begin{proof}
Assign a weight of $q(i)/M$ to every key $i$. For any key $i$, its
amortized time per access is $3(r(t)-r(x))+1=-3r(x)+1=-3\log(s(x))+1\leq
	-3\log(w(x))+1=O(\log(M/q(i)))$.
Since $W=1$, the net potential drop over the sequence is at most
$\sum_{i=1}^{N}\log(M/q(i))$. The result follows.
\end{proof}

Note that the time to perform the access sequence is equal to
$\O(M \cdot (1 + H(P)))$, where $H(P)$ is the entropy of the access probability
distribution defined as $P(i)=q(i)/M$. All static trees must take time
$\Omega(M\cdot (1+H(P)))$ for the access sequence. A proof of this lower
bound is available in \cite{information-theory}.
Since the taken for accesses in splay trees equals the asymptotic lower bound
for static search trees, they are \emph{statically optimal}, i.e., they are
as fast as any static search tree.

\vskip 1em

Let us denote the indices of accessed keys in an access sequence $S$ as
$k_1, \ldots k_M$ so that $S=(K_{k_i})_{i=1}^M$.

\begin{theorem}[Static Finger Theorem]
Given any chosen key $K_f$, the total access time is
$\O(N\log N + M + \sum_{j=1}^M \log(|k_j-f|+1))$.
\end{theorem}
\begin{proof}
	Proof included in \cite{splay}.
\end{proof}

As proven in \cite{dynamic-finger-1} and \cite{dynamic-finger-2},
accessing keys numerically close to recently accessed keys is also fast:
\begin{theorem}[Dynamic Finger Theorem]
The cost of performing an access sequence is
$\O(N+M+\sum_{i=2}^M \log(|k_i-k_{i-1}| + 1))$.
\end{theorem}

For any $j\in\{1,\ldots M\}$, define $t(j)$ to be the number of distinct
keys accessed since the last access of key $K_{k_j}$. If $K_{k_j}$ was not
accessed before, $t(j)$ is picked to be the number of distinct keys accessed
so far.
\begin{theorem}[Working Set Theorem]
The total access time is $\O(N\log N + M + \sum_{j=1}^M\log(t(j)+1))$.
\end{theorem}
\begin{proof}
	Proof included in \cite{splay}.
\end{proof}

The working set theorem implies that most recently accessed keys (the
``working set'') are cheap to access, which makes splay trees behave well when
the access pattern exhibits temporal locality.

To reiterate, the \emph{static optimality theorem} claims that splay
trees are as good as any static search tree. The \emph{working set theorem}
proves that the speed of accessing any small subset does not depend on
how large the splay tree is. By the \emph{dynamic finger theorem},
it is fast to access keys that are close to recently accessed keys in
the order of keys in $K$.
All of these results are implied by the unproven \emph{dynamic optimality
conjecture}, which proposes that splay trees are dynamically optimal binary
search trees: if a dynamic binary search tree optimally tuned for an access
sequence $S$ needs $\mathrm{OPT}(S)$ operations to perform the accesses, splay
trees are presumed to only need $\O(\mathrm{OPT}(S))$ steps.

It is conjectured that splay trees are as fast as any other kind of dynamic
search trees. A dynamic search tree algorithm $X$ is called $k$-competitive,
if for any sequence of operations $S$ and any dynamic search tree algorithm
$Y$ the time $T_X(S)$ to perform operations $S$ using $X$ is at most $k\cdot
T_Y(S)$. Splay trees were proven to be $\O(\log N)$-competitive, but
the dynamic optimality conjecture, which states that splay trees are
$\O(1)$-competitive, remains an open problem.

\section{Alternate data structures}
Since the dynamic optimality conjecture has resisted attempts at solving
for more than 20 years with little progress, a new approach was proposed in
\cite{tango}. The authors suggest building alternative search trees with small
non-constant competitive factors, and invented the \emph{Tango tree}, which
is $\O(\log\log N)$-competitive. However, worst-case performance of Tango trees
is not very good: on pathological sequences, they underperform plain BSTs
by a factor of $\Theta(\log\log N)$.
\cite{chain-splaying} extends splaying with some housekeeping operations
below the splayed element and proves that this \emph{chain-splay} heuristic
is $\O(\log\log N)$-competitive.

Finally, \cite{multisplay-trees} presents \emph{multi-splay trees}.
Multi-splay trees are $\O(\log\log N)$-competitive and they have all important
proven properties of splay trees. They also satisfy the \emph{deque property}
($\O(M)$ time for $M$ double-ended queue operations), which is only conjectured
to apply to splay trees.
Unlike Tango trees, they have an amortized access time of $\O(\log N)$.
While an access in splay trees may take time up to $\O(N)$, multi-splay trees
guarantee a more desirable worst-case access time of $\O(\log^2 N)$.

\section{Top-down splaying}
Simple implementations of splaying first find the node to splay, and then
splay along the traversed path. The traversed path may be either kept
in an explicit stack, which is progressively shortened by splay steps,
or the tree may be augmented with pointers to parent nodes.
Explicitly keeping a stack is wasteful, because it requires dynamically
allocating up to $N$ node pointers when splaying. On the other hand,
storing parent pointers makes nodes less memory-efficient, especially
with small sizes of keys.

A common optimization from the original paper \cite{splay} is
\emph{top-down splaying}, which splays while searching.
Top-down splaying maintains three trees while looking up a key $k$:
the left tree $L$, the right tree $R$ and the middle tree $M$.
$L$~holds nodes with keys smaller than~$k$ and $R$~holds nodes with keys
larger than~$k$. $M$ contains yet unexplored nodes.
Nodes are only inserted into the rightmost empty child pointer of $L$
and the leftmost empty child pointer of $R$. The locations of the rightmost
empty child pointer of $L$ and leftmost empty child pointer of $R$ are
maintained while splaying. Denote these locations $L_+$ and $R_-$.

Initially, $L$ and $R$ are both empty and $M$ contains the entire splay tree.

Every splaying step processes up to 2 upper levels of $M$. All nodes in these
levels are put either under $L_+$, or $R_-$ depending on whether their keys
are larger or smaller than $k$. $L_+$ and $R_-$ are then updated to point
to the new empty child pointers.
Finally, when $M$ has $k$ in the root, we compose the root of $M$ with its
two children, $L$, and $R$.

An implementation of top-down splaying in C by Sleator is available
at \url{http://www.link.cs.cmu.edu/link/ftp-site/splaying/top-down-splay.c}.
% TODO: Extend the description? Figure?

\section{Applications}
A drawback of splay trees is that accessing the $\O(\log N)$ amortized
nodes on the access path may incur an expensive cache miss for every node --
splay trees make no effort to exploit the memory hierarchy.
In contrast, lookups in B-trees and cache-oblivious B-trees
only cause up to $\Theta(\log_B N)$ cache misses, which is a~multiplicative
improvement of $\log B$.

On the other hand, splaying automatically adjusts the structure of the splay
tree to speed up access to recently or frequently accessed nodes. By the
working set theorem, accessing a small set of keys is always fast, no matter
how big the splay tree is. While B-trees and cache-oblivious B-trees are
also faster on smaller working sets, the complexity of any access remains
$\O(\log_B N)$, so accessing a~small working set is slower on larger
dictionaries.

Splay trees are frequently used in practice where access patterns are assumed
to be non-random. For example, splay trees are used in the FreeBSD kernel
to store the mapping from virtual memory areas to
addresses.\footnote{%
	See \texttt{sys/vm/vm\_map.h} in the FreeBSD source tree (lines 169--174
	at SVN revision 271244); available at
	\url{https://svnweb.freebsd.org/base/head/sys/vm/vm\_map.h?view=markup}.
}
Unfortunately, splay trees are much slower than BSTs on more random
access patterns: splaying always performs costly memory writes.
For some applications (e.g.,\ real-time systems), worst-case $\Theta(N)$
performance is also unacceptable (especially if the splay tree stores data
entered by an adversary).
