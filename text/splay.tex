\chapter{Splay trees}
Splay trees are a well-known variant of binary search trees introduced
by \cite{splay} that adjust their structure with every operation, including
lookups. Splay tree operations cost $\O(\log n)$ amortized time, which is
similar to common balanced binary search trees, like AVL trees or
red-black trees.
However, unlike balanced search trees, splay trees adjust their structure
according to the access pattern, and the running time of a sufficiently long
sequence of \textsc{Find} operations on a splay tree is within a constant
factor of an optimal static search tree.
Thus, splay trees perform better than balanced search trees on non-uniform
access patterns, which are common in real-world applications.
Additionally, the \textit{dynamic optimality conjecture} proposes that
the performance of splay trees on a sequence of operations (including updates)
is within a constant factor of any dynamic search tree tailored for
the particular operation sequence.

TODO: disadvantage - slow individual operations, lots of local adjustments

TODO: copied.

Splay trees are binary search trees. A binary tree contains nodes, each of which
contains one key $K$ and its associated value.
A node has up to two children - a left one and a right one.
The left subtree contains only keys smaller than $K$ and the right subtree
contains only keys larger than $K$.

% To find a key $K$ in a binary search tree, we use binary search:
% we compare $K$ with the key in the tree's root and continue to the subtree
% that may contain $K$. If the current node has no child for us to descend into,
% the tree doesn't contain $K$.

Splay trees use a heuristic called \textit{splaying}, which moves a specified
node to the root of the tree by performing a sequence of rotations along the
path from the root to this note. Every rotation costs $\O(1)$ time and preserves
the left-to-right order of nodes in the binary search tree.

\begin{figure}
\centering
TODO: figure jako splay-trees.pdf (see page 4)
\caption{Left and right rotation of the edge between $x$ and $y$.}
%\label{fig:rotation}
\end{figure}

Splaying proceeds up from a node $x$ until $x$ is the root by performing
the following \textit{splaying step}.

Splaying step:
Denote the splayed node by $x$, its parent by $p$ and its grandparent by $g$.

Case 1 (zig): If $p$ is the root, rotate the $px$ edge. (This case is terminal)

Case 2 (zig-zig): If $p$ is not the root and $x$ and
		$p$ are both left or both right children, rotate
		the $gp$ edge, then rotate the $px$ edge.

Case 3 (zig-zag): If $p$ is not the root and $x$ is a left child and $p$
		is a right child (or vice versa), rotate the edge $px$ and
		then rotate the edge now joining $g$ with $x$.

\begin{figure}
\centering
TODO: figure jako splay-trees.pdf (see page 5)
\caption{The three cases of a splaying step up to left-right symmetry --
	zig, zig-zig and zig-zag.}
%\label{fig:splay-step}
\end{figure}

Splaying a node $x$ of depth $d$ takes $\Theta(d)$ time, which is the same
as the time to access $x$ from the root.
% Splaying reduces the depth of every node along the access path by roughly
% one half. TODO: fakt? TODO: figure

To \textsc{Find} a key in a splay tree, we try to find it using the general
search algorithm for binary search trees. If the search succeeds, we splay
up the node containing the key. Otherwise, we splay up the last visited
node.
An \textsc{Insert} is similar - we find the right place for the new node,
insert it, and finally splay it up.

% TODO: make it so 
\textsc{Delete}s are slightly more complex. We splay up the node to delete.
If it has one child, we delete it and replace it by its child. If it has
two children, we replace its value by the value of the right subtree's
leftmost node. Then we delete the right subtree's leftmost node, which has at
most 1 child by definition.

All update operations on splay trees take time $O(\log n)$.

To analyze the performance of the \textsc{Splay} operation, let us assign
a fixed positive weight $w(K)$ to every key $K$. Define the size $s(x)$
of a node $x$ to be the sum of all $w(K)$ for all keys $K$ in the node's
subtree and define the rank $r(x)$ of $x$ to be $\log s(x)$.
For amortization, let us define the potential $\Phi$ of the tree to be
the sum of the ranks of all its nodes.
We will measure the time to splay a node in the number of rotations done
(or 1 if we didn't perform any rotations).

% TODO: to je lemma
\begin{lemma}
The amortized time to splay a node $x$ in a tree with root
$t$ is $3(r(t)-r(x))+1 = \O(\log(s(t)/s(x)))$.
\end{lemma}
\begin{proof}
If there were no rotations, the bound is immediate. Thus suppose there is at
least one rotation. Consider any splaying step.
Let $s$ and $s'$, $r$ and $r'$ denote the size and rank functions just before
and just after the step, respectively. We show that the amortized time for
the step is at most $3(r'(x)-r(x))+1$ in case 1 and at most $3(r'(x)-r(x))$
in case 2 or case 3. Let $p$ be the original parent of $x$ and $g$ the original
grandparent of $x$ (if it exists).

Case 1 (zig):
	We perform only one rotation between $x$ and $p$.
	This rotation may only change the rank of $x$ and $p$, so
	the amortized time for this step is $1 + r'(x) + r'(p) - r(x) - r(p)$.
	Because $r(p)\geq r'(p)$, the amortized time is $\leq 1+r'(x)-r(x)$.
	Finally, since $r'(x)\geq r(x)$, the amortized time is also
	$\leq 1+3(r'(x)-r(x))$.

Case 2 (zig-zig):
	We perform two rotations that may change the rank of $x$, $p$
	and $g$, so the amortized time for the zig-zig step is
	$2+r'(x)+r'(y)+r'(z)-r(x)-r(y)-r(z)$. The zig-zig step
	moves $x$ to the original position of $g$, so $r'(x)=r(g)$.
	Before the step, $x$ was $p$'s child, and after the step,
	$p$ was $x$'s child, so $r(x)\leq r(p)$ and $r'(p)\leq r'(x)$.
	Thus the amortized time for this step is at most $2+r'(x)+r'(g)-2r(x)$.
	We claim that this is at most $3(r'(x)-r(x))$, that is, that
	$2r'(x)-r(x)-r(g)\geq 2$.

	$2r'(x)-r(x)-r'(g)=-\log\frac{s(x)}{s'(x)}-\log\frac{s'(g)}{s'(x)}$.
	We have $s(x)+s'(g)\leq s'(x)$.

	The $\log$ function is concave, so
	if $a,b\geq 0$ and $a+b\leq 1$, $\log a + \log b$ is maximized
	by setting $a = b = \frac{1}{2}$, which yields the value $-2$.
	Thus the inequality above is correct and the amortized time is
	at most $3(r'(x)-r(x))$.

Case 3 (zig-zag):
	The amortized time of the zig-zag step is
	$2+r'(x)+r'(p)+r'(g)-r(x)-r(p)-r(g)$. $r'(x)=r(g)$ and $r(x)\leq r(p)$,
	so the time is $\leq 2+r'(p)+r'(g)-2r(x)$.
	We claim that this is at most $2(r'(x)-r(x))$, i.e.
	$2r'(x)-r'(p)-r'(g)\geq 2$. This can be proven
	similarly to case 2 from the inequality $s'(p)+s'(g)\leq s'(x)$.
	Thus the amortized time for a zig-zag step is at most
	$2(r'(x)-r(x))\leq 3(r'(x)-r(x))$.

By telescoping the sum of amortized time for all performed steps until
$x$ becomes the root $t$, we get an upper bound of $3(r'(t)-r(x))+1$
for the entire splaying operation.
\end{proof}

By choosing different assignments of $w(i)$, we can obtain several basic
results. Note that over any sequence of $m$ splay operations
the total potential $\Phi$ may only drop by up to $\sum_{i=1}^N \log(W/w(i))$
where $W=\sum_{i=1}^n w(i)$, since the size of the node containing $i$
is between $w(i)$ and $W$.

\begin{theorem}[Balance Theorem]
A sequence of $m$ \textsc{Find}s on an $n$-item splay tree takes time
$\O((m+n)\log n+m)$.
\end{theorem}
\begin{proof}
Assign $w(i) = \frac{1}{n}$ to every item $i$. The amortized
access time for any item is at most $3\log(n)+1$. We have $W=1$, so by
the previous observation the potential drops by at most $n\log n$ over
the access sequence. Thus the time for the access sequence is at most
$(3\log n+1)m + n\log n = \O((m+n)\log n + m)$.
\end{proof}

TODO: relate Balance theorem and costs of inserts and deletes

For any item $i$, let $q(i)$ be the number of accesses of $i$ in an access
sequence.
\begin{theorem}[Static Optimality Theorem]
If every item is accessed at least once, the the total access time is
$\O(m + \sum_{i=1}^n q(i)\log\frac{m}{q(i)})$.
\end{theorem}
\begin{proof}
Assign a weight of $q(i)/m$ to every item $i$. For any item $i$, its
amortized time per access is $3(r(t)-r(x))+1=-3r(x)+1=-3\log(s(x))+1\leq
	-3\log(w(x))=O(\log(m/q(i)))$.
Since $W=1$, the net potential drop over the sequence is at most
$\sum_{i=1}{n}\log(m/q(i))$. The result follows.
\end{proof}

TODO: dynamic optimality conjecture

Let us denote the indices of accessed items as $a_1, \ldots a_m$
in the order they occur, where $a_i \in \{1,\ldots n\}$.

\begin{theorem}[Static Finger Theorem]
Fix any item $i_f$. Then the total access time is $\O(n\log n + m +
\sum_{j=1}^m \log(|a_j-f|+1))$.
\end{theorem}
% Static finger theorem proof missing here.

For any access $j\in\{1,\ldots m\}$, let $t(j)$ be the number of different
items accessed since the last access of item $i_{a_j}$ or since the
beginning of the access sequence.
\begin{theorem}[Working Set Theorem]
The total access time is $\O(n\log n + m + \sum_{j=1}^m\log(t(j)+1))$.
\end{theorem}
% Proof missing here.

The balance theorem states that on a sufficiently long sequence, lookups in
the splay tree are as efficient as in any balanced search tree. The static
optimality theorem implies that the splay tree is as efficient as any fixed
search tree, including the optimum tree for the given sequence, since
it is known that any sequence of accesses on a binary search tree
takes time at least $\Omega(m+\sum_{i=1}^n q(i)\log(m/q(i)))$.
% TODO: cite: ABRAMSON, N. Information Theory and Coding. McGraw-Hill, New York, 1983.
The static finger theorem can be interpreted to mean that accessing keys
whose value is close to a recently accessed value is fast.
Finally, the working set theorem implies that most recently
accessed items (the "working set") are cheap to access, which
makes splay trees behave well when the access pattern exhibits temporal
locality.

\section{Implementation}


\section{???}
A drawback of splay trees is that accessing the $\O(\log N)$ amortized
nodes on the access path may incur an expensive cache miss for every node --
splay trees make no effort to exploit the memory hierarchy.
In contrast, lookups in B-trees and cache-oblivious B-trees
only cause up to $\Theta(\log_B N)$ cache misses, which is a multiplicative
improvement of $\log B$.

On the other hand, splaying automatically adjusts the structure of the splay
tree to speed up access to recently or frequently accessed nodes. By the
working set theorem, accessing a small set of keys is always fast, no matter
how big the splay tree is. While B-trees and cache-oblivious B-trees are
also faster on smaller working sets, the complexity of any access remains
$\O(\log_B N)$, so accessing a small working set is slower on larger
dictionaries.
