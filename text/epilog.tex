\chapter*{Conclusion}
\addcontentsline{toc}{chapter}{Conclusion}

Several data structures based on tries have also been proposed for the
dictionary problem. One example are \emph{y-fast tries} \cite{y-fast},
which allow \textsc{Find}s in time $\O(1)$ and updates and predecessor/successor
queries in time $\O(\log\log |U|)$, where $U$ is the key universe. They require
$\O(N \log\log |U|)$ memory.
The RAM model is required by y-fast tries -- they assume that RAM operations
can be performed on keys in constant time.
The original motivation for developing y-fast tries was lowering the memory
requirements of van Emde Boas queues ($\O(|U|)$) while keeping the same time
for operations.
% TODO: Try to find some practical numbers.

\emph{Judy arrays} are highly optimized dictionaries developed at IBM
\cite{judy-shop-manual, judy-patent}.
IBM sources claim that Judy arrays are very efficient in practice.
The data structure is based on sparse 256-ary tries with implementation tricks
that compress the representation.
Unfortunately, we found the existing documentation highly complex and sometimes
incomplete. On the other hand, an open-source implementation is available
at \url{http://judy.sourceforge.net/}, so it should be possible to independently
analyze the data structure and possibly to improve upon it.
