\chapter*{Conclusion}
\addcontentsline{toc}{chapter}{Conclusion}

Several data structures based on tries have also been proposed for the
dictionary problem. One example are \emph{y-fast tries}, which allow
\textsc{Find}s in time $\O(1)$ and updates and predecessor/successor queries
in time $\O(\log\log |U|)$, where $U$ is the key universe. They require
$\O(N \log\log |U|)$ memory.
% TODO: Can I start a sentence like this???
y-fast tries are built for the RAM model and they assume that RAM operations
can be performed on keys in constant time.
The original motivation for developing y-fast tries was lowering the memory
requirements of van Emde Boas queues ($\O(|U|)$ TODO: check) while keeping
the same time for operations.
% TODO: Try to find some practical numbers.

\emph{Judy arrays} are highly optimized dictionaries developed at IBM.
IBM sources claim that Judy arrays are very efficient in practice.
% TODO: Cite author & technical report
According to available documentation, the data structure is based on
sparse 256-ary tries with implementation tricks that compress
the representation.
Unfortunately, we found the existing documentation highly complex and sometimes
incomplete. On the other hand, an open-source implementation is available
at TODO, so it should be possible to independently analyze the data structure.

