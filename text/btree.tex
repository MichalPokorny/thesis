\chapter{B-trees}
\label{chapter:btree}
The B-tree is a very common data structure for storing sorted dictionaries.
B-trees are particularly useful where the external memory model is accurate,
because all information in every transferred block is actively used
in searching. For example, the ext4, Btrfs and NTFS filesystems use B-trees
to store directory contents and the MongoDB and MariaDB databases use B-trees
as indexes.  % TODO: Cite this claim.
Red-black trees, which are (in the left-leaning variant
described by \cite{left-leaning}) isomorphic to B-trees for $b=2$, are
commonly used to maintain in-memory dictionaries, for example in the GNU ISO
\Cpp Library as the implementation of the \texttt{std::map} STL template.
B-trees are also an optimal sorted dictionary data structure
for the external memory model assuming the only operation allowed on
keys is comparison.
B-trees were introduced in 1970 by \cite{btree}.

B-trees are a generalization of balanced binary search trees.
Nodes of a B-tree keep up to $2b$ key-value pairs in sorted order.
An inner node of a binary search tree with key $X$ contains two pointers
pointing to children with keys $< X$ and $\geq X$. In B-trees,
inner nodes contain $k\in[b;2b]$ keys $K_1\ldots K_k$. There are $k+1$
child pointers in internal nodes: one for every interval
$[K_1;K_2),\ldots [K_{k-1};K_k)$, plus one for $(-\infty;K_1)$ and
$(K_k;\infty)$ each.
Leaf nodes of a B-tree store $[b;2b]$ key-value pairs.

B-trees can store values for keys either within all nodes (internal or leaves),
or only in leaves. If values are stored only in leaves, keys stored
within internal nodes are only used as pivots, and may not necessarily
have a value within leaves -- if a key is deleted, it is removed from its leaf
along with its node, but internal nodes may retain a copy of the key.
Our implementation stores values only inside leaves to allow tighter packing
of internal nodes.

The essence of the $[b;2b]$ interval is that when we try to insert
the $(2b+1)$-th key, we can instead split the node into two valid nodes of size
$b$ and borrow the middle key to insert the newly created node into the parent.
Similarly, when we delete the $b$-th key, we first try to borrow some nodes
from neighbours to create two valid nodes of size $[b;2b]$, which works when
there are at least $2b$ keys in both nodes combined.
When there are $2b-1$ keys, we instead bring down one key from the parent
and combine the two neighbours with the borrowed key into a new valid node with
$b+(b-1)+1=2b$ keys.

In binary search trees, searching for a key $K$ is performed by binary search.
Starting in the root node, we compare $K$ with the key stored in the node
and we go left or right depending on the result. If the current node's key
equals $K$, we fetch the value from the node and abort the search.

The process of finding a key-value pair in a B-tree generalizes this
by picking the child pointer corresponding to the only range that contains
the sought key $K$. To pick the child pointer, we can either do a simple
linear scan in $\Theta(b)$, or we can use binary search to get
$\Theta(\log b)$ if $b$ is not negligible.
Once we reach a leaf, we scan its keys and return the value for $K$
(if such a value exists).

To \textsc{Insert} a key-value pair in a B-tree, we first find the right
place to put the new key-value pair, and then walk back up the tree.
If an overfull node (with $> 2b$ keys) is encountered, it is split
to two nodes of $\leq b$ keys. One of the keys from the split node
is inserted to the parent along with a pointer to the new node.
When we split the root node, we immediately create a new root node above the
two resulting nodes.

Deletions employ a similar procedure: if the current node is underfull
(i.e.,\ if it has $< b$ keys), it is merged with one of its siblings.
If there are not enough keys among the siblings for two nodes,
one of the siblings takes ownership of all keys and the other sibling
is recursively deleted from the parent. In this case, the corresponding
key from the parent is pulled down into the merged node.
The only exception is the root node, which may have $[1;2b]$ keys.
If deleting from the root would leave the root with only one child
(and zero keys), we instead make the only child of the root the new root.

For B-trees that store values within both internal nodes and leaves,
some deletions will need to remove a key used as a pivot within an internal
node. In this case, the deleted key needs to be replaced by the minimum of its
left subtree, or the maximum of its right subtree.
Since our implementation stores values only inside leaves, deletions will
always start at a leaf, which slightly simplifies them.

Thus, updates to the B-tree keep the number of keys of all nodes within
$[b;2b]$, so the depth of the tree is kept between $\log_2b N$ and $\log_b N$
and the \textsc{Find} operation takes time
$\Theta(\log b \cdot \log_b N)=\Theta(\log N)$ and $\Theta(\log_b N)$
block transfers. \textsc{FindNext} and \textsc{FindPrevious} also take time
$\Theta(\log N)$, but if they are invoked after a previous \textsc{Find}, we
can keep a pointer to the leaf containing the key and accordingly adjust it to
find the next or previous key, which runs in $\O(1)$ amortized time. Thus,
B-trees also allow scanning a contiguous range of keys of size $k$ in
$\Theta(\log_b N + k)$ time.

The \textsc{Insert} and \textsc{Delete} operations may spend up to
$\Theta(b)$ time splitting or merging nodes on each level, so they run in
time $\Theta(b \cdot \log_b N)$.

The main benefit of B-trees over binary trees is the tunable parameter
$b$, which we can choose to be $\Theta(B)$. Practically, if the B-tree
is stored in a disk, $b$ can be tuned to fit one B-tree node in exactly one
physical block. Thus, in the external memory model, \textsc{Find},
\textsc{Insert} and \textsc{Delete} all take $\O(1)$ block transfers
on every level, yielding $\Theta(\log_B N)$ block transfers per operation.

This is optimal assuming the only allowed operation
on keys is comparison: reading a block of $\Theta(B)$ keys will only
give us $\log B$ bits of information, since the only useful operation
can be comparing the sought key $K$ with every key we retrieved.
We need $\log N$ bits to represent the index of the key-value pair we found,
so we need at least $\log N/\log B=\log_B N$ block transfers.
