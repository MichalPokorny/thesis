\chapter{Implementation}
To measure the performance of various data structures, we implemented them
in the C programming language (using the C11 revision).
We have chosen this language for several reasons:
\begin{itemize}
\item The language is low-level, which enables a high degree of tuning by hand.
	Our data structures are also not slowed down by common features of
	high-level languages, such as garbage collection.
\item C toolchains, including the GCC compiler we used, are mature, and they
	can be expected to optimize code well.
\item The C language is very portable and most other languages can
	call C functions. This enables our data structures to be potentially
	easily used in other projects.
\end{itemize}
All dictionaries assume 64-bit keys and values (represented as the
\texttt{uint64\_t} type). It would be easy to allow keys and values of arbitrary
size, but such a choice would be likely to slow down operations on
the data structures, because a range of compiler optimizations would
be impossible in code assuming general lengths. For example, copying
keys and values would need to be a general loop or
\texttt{memcpy}/\texttt{memmove} call, while assuming a constant key/value
size of 64 bits lets us copy in one CPU instruction. Using the C++ language,
we could potentially use templates to make our data structures both general
and optimized. If we are willing to sacrifice some code style, we could
make the C code behave similarly by moving the implementation into a header
file, which could be configured to generate a general data structure
tuned to specific parameters prior to inclusion, as in the following example:
\begin{lstlisting}
struct my_value {
	uint64_t stuff[32];
};
#define IMPLEMENTATION_PREFIX myimpl
#define IMPLEMENTATION_KEY_TYPE uint64_t
#define IMPLEMENTATION_VALUE_TYPE struct my_value
#include "btree/general.impl.h"

void usage() {
	dict* btree;
	dict_init(&btree, &dict_myimpl_btree);
	// ...
}
\end{lstlisting}
We have decided to forgo generality to keep the code clean and readable.

\section{General dictionary API}
All implemented data structures can be used through a common API defined
in \texttt{dict/dict.h}. To encapsulate implementation details from
users of the API, we have used a common C "trick", where we represent
dictionaries as "virtual table pointers" and "private data".
The common API represents a dictionary as a pointer to an opaque structure.
The type pointed to is declared in \texttt{dict/dict.h} as:
\begin{lstlisting}[language=C]
typedef struct dict_s dict;
\end{lstlisting}

The \texttt{struct dict\_s} structure is defined in \texttt{dict/dict.c}
to prevent exposure of implementation details from users including
\texttt{dict/dict.h}:
\begin{lstlisting}[language=C]
struct dict_s {
	void* opaque;
	const dict_api* api;
};
\end{lstlisting}

The \texttt{opaque} pointer is maintained by the concrete implementation.
\texttt{dict\_api} is a "virtual method table" type defined in
\texttt{dict/dict.h}:

\begin{lstlisting}[language=C]
typedef struct {
	int8_t (*init)(void**);
	void (*destroy)(void**);

	int8_t (*find)(void*, uint64_t key,
			uint64_t *value, bool *found);
	int8_t (*insert)(void*, uint64_t key, uint64_t value);
	int8_t (*delete)(void*, uint64_t key);

	const char* name;

	// More function pointers.
} dict_api;
\end{lstlisting}

The \texttt{init} and \texttt{destroy} functions are passed a pointer to the
\texttt{opaque} pointer and they respecively initialize and deinitialize it.
The \texttt{name} field is a human-readable name of the data structure.
Other fields are given the \texttt{opaque} pointer as the first argument
and they act as implementations of their respective general versions operating
on general dictionaries declared in \texttt{dict/dict.h}:

\begin{lstlisting}[language=C]
int8_t dict_find(dict*, uint64_t key,
		uint64_t *value, bool *found);
int8_t dict_insert(dict*, uint64_t key, uint64_t value);
int8_t dict_delete(dict*, uint64_t key);
\end{lstlisting}

Dictionary data structures may optionally implement successor and predecessor
lookups. The public API consists of the functions \texttt{dict\_next} and
\texttt{dict\_prev}. If the data structure does not implement these operations,
it may set the appropriate fields in \texttt{dict\_api} to \texttt{NULL},
in which case the public functions will return an error and do nothing.
\begin{lstlisting}[language=C]
int8_t dict_next(dict*, uint64_t key,
		uint64_t *next_key, bool *found);
int8_t dict_prev(dict*, uint64_t key,
		uint64_t *prev_key, bool *found);
\end{lstlisting}

Every data structure implementation exposes a global variable of type
\texttt{const dict\_api} named \texttt{dict\_mydatastructure}. For example,
a B-tree can be used as follows:

\begin{lstlisting}[language=C]
#include <inttypes.h>
#include <stdio.h>

#include "dict/btree.h"
#include "dict/dict.h"

void example_btree() {
	dict* btree;
	if (dict_init(&btree, &dict_btree) != 0) {
		printf("Error initializing B-tree.\n");
		return;
	}
	if (dict_insert(btree, 1, 100) != 0) {
		printf("Error inserting 1 -> 100.\n");
	}
	// ...
	if (dict_find(btree, 42, &value, &found) != 0) {
		printf("Error looking for key 42.\n");
	} else {
		if (found) {
			printf("Found 42 -> \%" PRIu64 ".\n",
					value);
		} else {
			printf("No value for 42.\n");
		}
	}
	// ...
	if (dict_delete(btree, 1) != 0) {
		printf("Failed to remove key 1.\n");
	}
	dict_destroy(&btree);
}
\end{lstlisting}

Finally, we have decided to reserve one key for internal purposes.
This key is defined in \texttt{dict/dict.h} as the macro
\texttt{DICT\_RESERVED\_KEY} and its value is \texttt{UINT64\_MAX}.
Inserting this key into a dictionary or deleting it will result in an error.
Reserving this value is useful for representing unused slots in B-tree
and $k$-splay tree nodes.

We have implemented the following dictionary data structures:
\begin{itemize}
\item \texttt{dict\_cobt}, declared in \texttt{dict/cobt.h}:
	a cache-oblivious B-tree as described TODO.
\item \texttt{dict\_splay}, declared in \texttt{dict/splay.h}:
	a splay tree.
% TODO: btree, ksplay, htable
\end{itemize}

\section{Performance measurement}

TODO: RAM
TODO: perf
